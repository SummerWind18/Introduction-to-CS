\documentclass{article}
\usepackage[UTF8]{ctex}
\usepackage{geometry}
\usepackage{multirow}
\usepackage{natbib}
\usepackage{graphicx}
\usepackage{setspace}
\usepackage{enumerate}
\usepackage{caption2}
\usepackage{datetime}

\pagestyle{plain}
\geometry{left=3.18cm,right=3.18cm,top=2.54cm,bottom=2.54cm}
\renewcommand{\today}{\number\year 年 \number\month 月 \number\day 日}
\renewcommand{\captionlabelfont}{\small}
\renewcommand{\captionfont}{\small}

\begin{document}

\begin{figure}
    \centering
    \includegraphics[width=8cm]{upc.png}
    \label{figupc}
\end{figure}

\begin{center}
	\quad \\
	\quad \\
	\heiti \fontsize{45}{17} \quad \quad \quad
	\vskip 1.5cm
	\heiti \zihao{2} 《计算科学导论》个人职业规划
\end{center}

\vskip 1.7cm

\begin{quotation}
	\doublespacing
    \zihao{4}\par\setlength\parindent{7em}
	\quad

	学生姓名:\underline{\quad \qquad \ 尚博文 \ \qquad \quad}

	学\hspace{0.6cm} 号:\underline{\qquad \ 2007010318 \ \qquad}
		
	专业班级:\underline{\qquad \ 计算2003 \ \qquad  }
		
    学\hspace{0.6cm} 院:\underline{计算机科学与技术学院}

	\vskip 2cm
	\centering
	\begin{table}[h]
        \centering
        \zihao{4}
        \begin{tabular}{|c|c|c|c|c|c|c|c|c|}
            \hline
            \multicolumn{5}{|c|}{分项评价} &\multicolumn{2}{c|}{整体评价}  & 总    分 & 评 阅 教 师\\
            \hline
            自我 & 环境 & 职业 & 实施 & 评估与 & 完整性 & 可行性 &\multirow{2}*{} &\multirow{2}*{}\\
            分析& 分析& 定位 & 方案 & 调整 & 20\% & 20\% & ~&~ \\\
            10\% & 10\% & 15\% & 15\% & 10\% & &  &~ &~\\
            \cline{1-7}
            & & & & & & & ~&~ \\
            & & & & & & & ~&~ \\
            \hline
        \end{tabular}
    \end{table}
    \vskip 1.7cm
    \today
\end{quotation}

\thispagestyle{empty}
\newpage

\setcounter{page}{1}

\section{自我分析}

\subsection{自然条件}

\begin{itemize}
    \item 性别:男
    \item 年龄:19
    \item 身体条件:良好
    \item 健康状况:良好,但运动较少
    \item 居住城市:家乡为新疆吐鲁番,现居青岛
\end{itemize}

\subsection{性格分析}

本人品行端正、善良且有社会责任感。性格相对而言的外向,能够很好的处理人际关系,身处三四个相处较好的朋友圈,交友广泛并且拥有一定数量的深交朋友。脾气不错,有足够的耐性。性格多样,在不同类型的朋友面前表现不同的自己。善于独立思考,注重团体力量,善解人意,做事全力以赴,有耐性,刻苦,实际而热情,意志坚定、有毅力。

有时易冲动,情绪欠稳定一波动,偶尔会有消极情绪,并持续较长时间。没有方向感与目标,感到茫然。对没有兴趣的东西无法坚持,只能学好感兴趣的东西。

渴望能力能受到肯定,希望能有不受条规限制与拘束的发展空间,乐意与正直却又懂得变通、有社会责任感与自身自责任感、上进心的人交往。拥有正确的世界观、人生观、价值观,并通过在大学的学习,使自己具备良好的的价值倾向。

\subsection{教育与学习经历}

小学至高中一直在家乡吐鲁番度过,学习对我来说从不是一件痛苦的事情。由于学习、竞争压力较小,加上自身良好的学习习惯,我的学习成绩一直保持优良,也让我成为同龄人中佼佼者。由于父母工作的原因,早在幼儿园时我便接触到了计算机,感受到了计算机的魅力,直到现在也对计算机热情不减。

因为较早接触计算机,电子游戏也是接触了不少,不知不觉的锻炼了英语方面的能力。初中因为对ACT和RPG游戏非常喜爱,抄下了许多全英文游戏剧本,通过自学以及为游戏人物配音,极大地提升了英语口语能力以及写作能力。此后也一直对英语有极大兴趣,通过各种网站平台看视频学英语,掌握了很多课外知识,加上本身对英语的热爱,英语成绩遥遥领先。

\subsection{工作与社会阅历}

我参加过一些社会实践活动。曾帮助某公司分发传单,帮助社区工作人员处理居民信息收集,疫情期间作为社区志愿者为小区居民服务,帮助其他同学疏导心理问题等。这些活动帮助我提升自己的人际交往能力,很好地促进了自身成长发展。

\subsection{知识、技能与经验}

对C++有一定了解,能较好的使用C++编程语言并解决实际问题,接触过Python、Java等其他编程语言,目前正在学习各种竞赛算法知识,数据结构等。

懂得如何使用各种办公工具,如Office、Letex、Markdown等,会使用CSDN、Github、等论坛网站查阅资料,在中国知网、校图书馆搜集文献资料。有较强的自主学习能力,主动学习自己想学、感兴趣的知识。

加入学校ACM俱乐部后,成为了后备营的一员,通过近一学期的努力,积极学习程序设计相关知识以及算法竞赛知识,参加学校组织的训练赛,目前为20级ACM现役队员之一。


\subsection{兴趣爱好与特长}

\begin{itemize}
    \item 学习过跆拳道,获得绿带,省三等奖
    \item 喜欢英语,口语较好,能熟练运用于生活
    \item 喜欢羽毛球,乒乓球等球类运动
    \item 喜欢听纯音乐,古典乐,英文歌
\end{itemize}

\section{环境分析}

\subsection{社会环境分析}

当前社会的总体就业形势已经备受瞩目,由可知的数据显示,大学生的就业将要面临越来越大的压力,如若经济增长拉动就业的能力降低,用人需求明显减少,供求缺口将进一步加大;社会学家分析,就业形势需要集中解决的突出难点之一首先集中在毕业生就业难。在大学毕业生面临越来越严峻的就业形势的同时,许多的毕业生缺乏自主择业观,缺乏对自身的评估也成为了大学生就业难的主要原因之一。由此,我们应该在平时的学习之余,多关注社会,从而使得自己能够根据社会需要,去更好的练就自己的能力,全面提高自己的综合素质;除此之外,更需要结合当前的社会形形势、分析自己毕业后的社会形势,提前做好自己的职业规划,对自己进行充足的评估。为自己将来要实现的目标,做一个客观、完整、全面的规划。这也是这篇职业生涯规划的目的所在。

\subsection{家庭环境分析}

我所生活的家庭温馨和睦,父母均为个体经营,家庭有较稳定的经济收入。家人期望我能够对计算机科学方面做出一定贡献,偏向于从事与计算机应用相关的工作。但父母尊重我的发展选择,尊重我个人的职业意愿。

\subsection{职业环境分析}

首先,全球现处于第三次工业革命, IT行业在其中举足轻重;且现在也是“三步走”战略”和“新三步走”战略的关键时期。这一时期定会仍很大的机遇和契机,我们正处于这一伟大的时机。我国在经过三十年改革开放,综合国力得到了显著的提升。

现正暴发新一轮的世界性的经济危机,这对我们是挑战更是难得的机遇,到我们走向社会时更能一展所能。随着成功举办、参加了多次国际性的活动,更是使我国成为国际性的大国。2000年“网络泡沫”破灭后,全球IT人才需求缩水,但世界IT产业的生产基地和研发中心正陆续在我国设立。全球新一轮的产业转移正推动我国由IT大国向IT强国转变,这将进一步刺激国内对计算机专业人才的'需求。
 
现在我国的IT行业人才不足且有严重的结构性的失衡,但其中也更重要的是高技术人员的需求。社会需要的更多的是高技术性的IT人才,用人单位更是提高这方面的门槛。现在的计算机已经得到了极广的普及,各高校都很重视这方面的培养,企业也重视培训。在大学生就业形势危机的情形下,IT行业也是日趋激烈,但也仍会是“抢手贷”。

\subsection{地域与人际环境分析}

北京市是我国的政治、文化、科技创新、国际交往中心,同时也是国家中心城市、超大城市和现代化国际城市,是我国四个直辖市之一。

北京位于华北地区,地势西北高、东南低,三面环山,气候主要以北温带半湿润大陆性季风气候为主,夏季高温多雨,冬季寒冷干燥,春秋时期较为短促。它一座有着三千多年历史的古都,拥有着悠久的历史文化。

作为国家的首都,北京将主要承担政治中心的工作,同时兼顾其他领域的发展。北京是全国教育最发达的地区,拥有世界第三大图书馆中国国家图书馆,聚集了全国数量最多的重点高校,具有雄厚的科技实力和巨大的发展前景。

且北京云集许多互联网公司总部和研发中心,就业岗位多,诸多领域的大牛均生活在这座城市,人际环境良好,但存在房租食宿贵、交通较为拥堵等问题。

\section{职业定位}

\subsection{行业领域定位与理由}

基于上述的了解与分析,我认为我适合软件研发相关的工作,理由如下:

第一,软件工程师现在缺口非常的大,一项来自中华英才网的统计数据显示:软件工程师需求量每年的缺口超过60万人,而且这个数据随着中国软件的普及而快速递增。因为软件工程师这么紧缺,所以合格的软件工程师的待遇非常的好。软件工程师起点很高,在企业里处于两高地位(薪水高、地位高)。另外,软件工程师的发展空间非常的大,软件工程师可以做数据库工程师、软件安全工程师、软件管理员,随着经验的积累可以做高级网路工程师、项目主管、项目经理,如果有魄力的话,还可以自己创业。总之,软件工程师的发展空间非常的大。

第二,我本身对程序设计有着浓厚的兴趣。我爱编程,我觉得软件开发是一个很酷的职业。主要原因如下:

1、创造性

如果你问别人创造性的工作有哪些,别人通常会说像作家,音乐家或者画家之类的职业。但是极少有人知道软件开发也是一项非常具有创造性的工作。我认为它是最符合创造性定义的了,从根本上说,编程就是为问题创造解决方案。在一天结束的时候,你创造了一些当天早上并不存在的东西,不得不说这是一种创造事物的纯粹的快乐。

2、协作性

有人认为程序开发就是程序猿们独自坐在他们的电脑前,然后撸一天的代码。但是事实上软件开发通常总是一个团队努力的结果。我想我的性格适合与人交流沟通,合作共进。我享受大家一起为了一个目标努力,提出自己的意见看法,最终得到解决方案这样的过程。

3、高需性

世界上越来越多的人在用软件,正如 Marc Andreessen 所说 " 软件正在吞噬世界 "。虽然程序猿现在的数量非常巨大,但是,需求量一直处于供不应求的局面,正如之前所说。软件公司最大的挑战之一就是找到优秀的程序猿。

4、高酬性

软件开发可以带来不菲的收入。这个事实与对程序猿的高需求意味着收入相当可观。当然还有许多更捞金的职业,但是相比一般人群,我认为软件开发者确实“日进斗金”。

5、前瞻性

许多工作岗位正在慢慢消失,往往是由于它们可以被计算机和软件代替。但是所有这些新的程序依然需要开发和维护,因此,程序猿的前景还是相当好的。


\subsection{职业岗位起点定位与理由}

我认为,在学业结束后,我会成为一名企业的软件研发工程师,理由如下:

第一,我将在大学期间学习除了最基础的编程语言(C/C++/JAVA等)、数据库技术(SQL/ORACLE/DB2等)之外,如JAVASCRIPT、AJAX、HIBERNATE、SPRING等前沿技术,并会了解一些关于网络工程和软件测试的其他技术。此外,我会学着做一些项目,积累开发经验。

第二,软件研发工程师偏向于实践,是上升到理论研究层面的“跳板”。在企业的工作经验能让我更快的熟练运用所学知识,提升自己的实践能力,对理论知识有更加深入的理解。

\subsection{职业目标与可行性分析}

\subsubsection{短期目标}

完成学业,期间参加ACM比赛,争取拿到较好的成绩,对软件开发方面有更加深入的理解。

\subsubsection{中长期目标}

中期目标成为一名软件开发工程师,在企业从事软件开发相关方面的工作,为企业创造价值,为自身谋取利益。

\subsubsection{可行性分析}

第一,对于程序设计方面我有着浓厚的兴趣,我具有坚持下去的热情和动力。

第二,我的创造性的思维特点适合软件开发相关工作,我的持之以恒的性格特点适合从事科学研究。

第三,评估与调整和老师的指导、工作的经历可以帮助我逐步实现目标。

\section{实施方案}

\subsection{程序设计竞赛}

通过程序设计竞赛及其日常训练来不断提升自己的算法能力和编程能力,具体方案如下:

\begin{itemize}
    \item 每月学习至少一项新算法,并进行一定练习至掌握程度
    \item 积极参加ACM训练,在不断练习中查漏补缺,提升算法熟练度
    \item 提升对思维题的思考与探索能力,弥补自身考虑不周全的缺点
    \item 通过程序设计竞赛结交竞赛圈和与之相关企业圈的伙伴与朋友
\end{itemize}

\subsection{日常生活学习}

通过计划来提升日常学习生活质量,增强自己的人际交往能力进而发展人脉,缓解并释放工作压力,保证自己的身心健康,具体方案如下:

\begin{itemize}
    \item 认真学习每一门课程,重视每一项课程作业和考试考核
    \item 积极参加“石光”活动,发展人脉关系,释放学习压力
    \item 积极参加体育锻炼,保证身心健康
\end{itemize}

\subsection{其他}

根据实际情况在评估与调整阶段动态调整每一学期、每一学年的计划,更加纵容的向着实现个人职业规划的目标砥砺奋斗,不断前行。

\section{评估与调整}

\subsection{评估时间}

毕业前,每学期进行一次简要的评估与调整,适当调整下学期计划,每学年进行一次较为详细的评估与调整,适当调整下学年计划。

毕业后,每半年进行一次评估与调整,适当制定符合实际的月度计划和季度计划,并根据实际情况和期望目标的差异进行合理分析、适当调整。

\subsection{评估内容}

\begin{itemize}
    \item 评估是否完成了阶段性的计划和目标,量化成果与期望值
    \item 评估自己的科研能力和技术能力是否较以前有所进步,是否学习到新的知识
    \item 是否拥有一个小团队钻研某一难题或开发某一项目
    \item 随时间的发展评估评估内容的合理性和完整性,进而改善评估内容
\end{itemize}

\subsection{调整原则}

根据自身情况与环境变化进行合理调整,需要考虑具体实施方案的可行性和目标可达性,进而分步设立目标。

\end{document}
